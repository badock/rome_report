\chapter{Collaboration between Nova controllers}
\label{sec:collaboration}



\section{Organisation of the collaboration between components of OpenStack}

Collaboration between components of OpenStack is based on three different ways:
AMQP \footnote{AMQP: Advanced Messaging Queuing Protocol} bus, APIs and
sharing of inner-states via a replicated relational databases. Each of the 
preceding ways of collaborating have specific purposes: some are intended for 
local collaboration as with sub-services of a controller, while some of these
ways are intended for inter-controller collaboration.

\subsection{APIs: Exposing services to enable Remote Procedure Call (RPC)}

\begin{figure}[htbp]
        \centering
        \includegraphics[width=14cm]{figures/NovaServices_API.pdf}        
        \caption{Collaboration between service thanks to APIs.}
        \label{fig:sequence_vm_creation}
\vspace*{-.3cm}
\end{figure}




APIs can be requested through two means: production/consumption of message on an
AMQP bus or HTTP request.

\subsubsection{AMQP bus: making sub-services collaborating with shared queues}

The AMQP bus is used to connect services of a same controller via exchange of
message. The message are send in queues, and then consumed by components that
are registered to the queue. In the case of the Nova service, its sub-services
are plugged on a RabbitMQ bus, and exchange data on it. However, post-mortem
study of logs showed that the Queue was not important for inter-nodes
collaboration: in a configuration where each controller has his own queue
isolated the others', collaboration was still possible. queue seems to be a mean
to do RPC \footnote{RPC: Remote Procedure Call} between services.


\subsubsection{HTTP request}

Services of OpenStack can also expose some of their functions via web service:
it is possible for an user that is leveraging KeyStone authentication to query
these web services to access service's APIs.

\subsection{Shared inner-states: collaboration between remote controllers}


\section{Creation of a VM: detailed workflow}

During the study of OpenStack components, it was noticable that a large part of
the collaboration that happend between controllers during the creation of a VM
\footnote{VM: Virtual machine} was done via the sharing of inner-states.

\begin{sidewaysfigure}[htbp]
        \centering
        %http://www.websequencediagrams.com/?lz=dGl0bGUgQ3JlYXRpb24gb2YgYSBWTQoKVXNlciAtPiArY29uZHVjdG9yLm1hbmFnZXIqXG5AZWNvbm9tZTE1OiBidWlsZF9pbnN0YW5jZSgpCgATHgBPBXNjaGVkdWxlci5maWx0ZXJfAAgJAE4PACAJX3J1bgBZDAAbJwCBOwdtcHV0ZQCBLRM3OiAASQ8KIyBCdWlsZCAAgUkICgAjGy0-AD4eXwCCCxEKICAAKiZyZXNzb3VyY2VfdHJhY2tlcgCBKA0AgmYIX2NsYWltKCkATAsAHRwtPi0AgW8dAEYIAH0pAIFgFmFsbG9jYXRlX25ldHdvcmsAgRsFAIFZIAAlBwCDDxYASQlmb3IAhF8MICAgIAAeGy0-ADAmbWFjX2FkZHJlc3MAC0pmaXhlZF9pcABmQXVwZGF0ZV9kbgB4JQCDRB4AgwsHX2luZm8AhGwgAIQPHQCFKSEAhEweX3ByZXBfYmxvY2tfZGV2aQCDRwcAhjA8c3Bhd24AhEYldmlydC5kcml2AIZHDwA2DAAOFi0AhiIgb2s&s=default
        \includegraphics[width=23cm]{figures/sequence.png}        
        \caption{Services involved in the creation of a VM.}
        \label{fig:sequence_vm_creation}
\vspace*{-.3cm}
\end{sidewaysfigure}

Indeed, when a request for the creation of a VM was made, the \emph{nova-api}
sub-service performed actions that resulted in the storage of an VM object in
the database. A simplified view of this VM object can be found in the following
listing:

\begin{lstlisting}[language=json,firstnumber=1]
{
   'hostname':u'vm1',
   'vm_state':u'building',
   'progress':0,
   'project_id':u'1b46e91b72984e47ad0d8d127f59e978',
   'launched_at':None,
   'scheduled_at':None,
   'deleted':0,
   'key_name':None,
   'task_state':u'scheduling',
}
\end{lstlisting}

It is noticable that this object will evolve during the creation of the VM: at
each step of the creation, a service will update some of the fields. To hand
over to a another service in a next step, there is two possibilities:

\begin{itemize}
\item Directly contact the next service
\item Push an updated version of the object in the database and let a service
poll it and start the next step of the VM creation.
\end{itemize}